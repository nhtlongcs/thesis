\begin{EnAbstract}
There is no doubt that artificial intelligence is one of the most talked-about topics of recent interest. And within the scope of AI, a more popular area is its application to healthcare – or more specifically, to extracting information from medical images. This is an urgent problem, as \textbf{volumetric CT images} can contain a great deal of valuable information about patients' health. But accurately extracting that information requires sophisticated image processing techniques – which is where AI comes in.

In this thesis, we focus on the problem of \textbf{volumetric CT images segmentation}: specifically, \textbf{organ segmentation}. We investigate two different approaches to this problem: \textbf{semi-supervised learning} and \textbf{active learning}. Especially, our method is an improvement of the traditional 2D approach, which is a concept with two components including \textbf{references} and \textbf{propagation}: references would be supporting the expert and give preliminary results, and propagation will be from the results referenced and predicted across the slices that have partial similarity in terms of slices. The reason behind that is the model could keep the resolution without any complex preprocessing methods; The proposed method has a high adaptive quality, which is proven by the reusability and \textbf{unlimited classes prediction ability} from our implementation application.

To use the full potential of unlabelled data, we approach it in two ways as follows. In the references process, we use semi-supervised by designing training models and tactics based on \textbf{Cross Pseudo Supervision} strategy, which helps capture information at a general level. In the propagation process, we use the same type of method to force the model to learn visual similarity and domain-specific features, this positive result can be applied in many types of parts or unseen contexts.

\pagebreak

In medical images, slices have different importance and influence, for example in CT volume images, slices that are centered and contain more classes will carry more information. For that reason, we propose a training strategy that prioritizes using \textbf{active learning} to generate \textbf{pseudo labels} with high confidence, which is used to suggest potentially usable unlabelled samples to contribute to training a better model.

Our results show that both methods are effective at accurately identifying organs in CT scans. The result is boosted \textbf{from 0.5 to 0.78} and comparable with state-of-the-art methods using 3D approaches. Our method using semi-supervised using active learning outperforms traditional 2D method by a significant margin. Furthermore, we also implement an application to demonstrate the usability and practicality of the thesis.
\end{EnAbstract}
