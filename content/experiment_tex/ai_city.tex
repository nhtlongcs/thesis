\section{AI City Challenge 2021}
Natural language querying offers an efficient way to retrieve vehicle track from traffic camera. 
The AI City Challenge 2021 launched a new track to push the envelop of research and deployment in this new topic. 
The workshop provides the CityFlow-NL benchmark and an evaluation system for participants to test their approaches on the real-world scenario.

\subsection{Dataset}
The CityFlow-NL benchmark contains 3028 video tracks of 666 target vehicles, captured from 40 calibrated cameras and 5,289 unique natural language descriptions. 
The dataset is splitted into training set with 2498 video-query pairs and private test set with 530 pairs.
In this work, we utilize the AI City Challenge platform to evaluate our analytics and proposed solutions.

\subsection{Evaluation metrics}
The challenge leverages the standard metrics for retrieval tasks. 
\begin{itemize}
    \item Evaluation of ranked retrieval set Mean Reciprocal Rank (MRR) \cite{voorhees1999trec}.
    \begin{align}
        \mathrm{MRR} = \frac{1}{|Q|}\sum_{i=1}^{|Q|}\frac{1}{\mathrm{rank}_i}
    \end{align}
    where $Q$ is the set of text queries and $\mathrm{rank}_i$ is the rank of the right track for each query in $Q$.
    \item Evaluation of unranked retrieval set Recall at K:
    \begin{align}
        \mathrm{R@K} = \frac{\mathrm{TP}@K}{\mathrm{TP}@K + \mathrm{FN}@K}
    \end{align}
    Where $\mathrm{TP}@K$ denotes the number of queries whose right tracks place in top-$K$. $\mathrm{FN}@K$ are the others with out-of-top results.
\end{itemize}