\section{Future works}

It is undouted that there is a lot of space for our method to improve. Although it is more resource-efficient than the original method, it requires huge memory bank to store information during the testing phase. There have been many recent research with solution to alleviate this problem by using multiple memory storage mechanism \cite{cheng2022xmem}, or an identity assignment bank to associate multiple objects at once \cite{yang2022associating}. Unfortunately, at the time they are introduced, we have nearly come to the end of our thesis. It would be highly recommended for readers to look into these research.

Some drawback of the model comes from the unique property of the dataset, however we lack of expert knowledge to carefully look at those error case by case. It is encouraged in the future for researchers to perform in-depth investigation to obtain more valuable insights.