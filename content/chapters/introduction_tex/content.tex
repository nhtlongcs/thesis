\section{Thesis content}
\label{sec:thesis_content}

This thesis is structured into 7 chapters:

\textbf{Chapter 1}
In chapter 1, we present about the 3D CT volume organ segmentation task and its applications for many medical fields such as image-guided surgery, radiation therapy planning, and 3D printing. The purpose of this chapter is to provide an overview of the 3D CT volume organ segmentation problem and variants scenarios. We first discuss the problem statement and challenges involved in 3D CT volume organ segmentation. Next, we describe some key applications that can benefit from accurate 3D organ segments especially in interactive scenario. After that, we demonstrate our proposed approach overview for efficiency and accurately extracting organs from 3D CT volumes. Finally the thesis structure that we present will be overviewed.

\textbf{Chapter 2}
In chapter 2, we discuss all necessary background knowledge related to our work. First of all, we start with the very beginning concept in machine learning and then introduce some fundamental models used in processing complex data such as time-series or digital images. Then we introduce a list of Computer Vision problems which play an important role in our proposed approach. 

\textbf{Chapter 3}
In chapter 3, we introduce the volume organs segmentation problem, inspire from video object segmentation problem and is most related to our work. We also discuss the semi-supervised approaches which are widely used to solve medical image segmentation task and details about the state-of-the-art model that we utilized in our system. Next, we introduce related works which have a same scenario and motivation to solve the interactive segmentation concept. We provide detailed discussions about the Interactive methods and Positional encoding technique which applied directly to our propose architecture.

\textbf{Chapter 4}
In this chapter, we present our solution for tackle the interactive volume object segmentation. Our proposed approach include reference and propagation module. At first, the pipeline of the method is shown. Then, each module is introduced and the detailed implementation is presented. We apply the distillation technique not only in architecture design but also in the training strategy stage. Finally we dive deep in the potential of loss function especially for medical.

\textbf{Chapter 5}
In this chapter, we describe the experiment dataset, evaluation metrics, and challenge platform. Besides, we provide the details configurations and implementation details of our proposed method. Based on that, we also analyze the result with on our observations.

\textbf{Chapter 6}
In this chapter, we present the applications of our proposed method for interactive video volume segmentation, including a annotation tool. 

\textbf{Chapter 7}
In this chapter, we report the results and discuss about the future works for improving our proposed method and its applications. In our thesis, we proposed a novel pipeline to tackle the problem of CT volume organ segmentation along with an interactive tool that can help labeling process become less complicated. Overall, through ablation study, our method shows improvements over the original baseline, yet still have weaknesses. If these shortcomings can be resolved, we believe it can bring many breakthrough for both the deep leaning and medical fields. Having said that, it is left for the future works. 
