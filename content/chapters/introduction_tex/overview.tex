\section{Overview}
\label{sec:overview}
% Organs segmentation có vai trò quan trọng trong các bước xử lí lâm sàng (clinical applications) vì ảnh hưởng đến các yếu tố như phát hiện bất thường trong cơ quan, chẩn đoán bệnh, etc. Tuy nhiên việc định lượng các cơ quan một cách chính xác khá là expensive và tốn thời gian, chính vì vậy một giải pháp cần nhanh chóng và ít tốn công sức hơn cần ra đời để giải quyết vấn đề trên. Học máy đã phần nào giúp con người ở các tác vụ fully supervise và các bài toán dữ liệu lớn. Tuy nhiên thì ở trong bài toán này, chúng ta cần phải đưa ra giải pháp có độ thích nghi nhanh chóng, và chỉ ở vai trò hỗ trợ bác sĩ tốn ít công sức hơn thay vì thay thế hoàn toàn khi thực hiện tác vụ. Semi supervise là một giải pháp thay thế tiềm năng được đề xuất gần đây.
\lipsum[1]

\subsection{CT volume segmentation}
\lipsum[2]
% 1. 
% Mục tiêu của thị giác máy tính là bắt chước khả năng understanding của con người, dựa vào những phát triển lớn mạnh theo cấp số nhân của deep learning hiện nay, tham vọng con người dần thay đổi, đó chính là bắt chước khả năng hiểu biểt của một expert trong một domain, lĩnh vực nào đó. 
% Understanding là gì?
% hiểu biết có nghĩa là xác định kiến thức có thể là gì thu được từ dữ liệu trực quan đã cho, các đối tượng bên trong và mối quan hệ giữa các đối tượng đó. Understanding về mặt thị giác đối với một cá thể expert là điều có thể làm một cách dễ dàng vì họ có thể ngay lập tức nhận thức và phân loại khái niệm phức tạp dựa trên prior knowledge. Khi một hình ảnh được hiển thị, con người có thể xác định và phân tích các đối tượng trong đó. Chính vì sự không rõ ràng và khó diễn giải, tác vụ này đã được chứng minh là một vấn đề cực kỳ thách thức đối với một máy tính.
% Nói về tính challenge của segmentation
% Trong các tác vụ cơ bản của computer vision field, classification, detection, segmentation, nhiệm vụ segmentation, yêu cầu cao nhất về chi tiết, yêu cầu hệ thống cung cấp chú thích, cấp pixel cho hình ảnh. Đủ để diễn giải về ngữ cảnh, mối quan hệ cũng như đối tượng, bao hàm hầu như toàn bộ 2 tác vụ còn lại để giải quyết bài toán segmentation. Segmentation đặc biệt hơn ở mức có thể phát hiện vật thể bị chồng chéo tránh trường hợp ambiguous, tuy nhiên đối với detection chỉ có thể đưa ra kết quả ở mức abstract hơn là vùng bounding của vật thể đó.
% Volume object segmentation là tác vụ tạo phân đoạn cấp pixel bằng cách chia volume thành các lớp (slices) và segment vùng của các đối tượng và được sử dụng để thu thập thông tin từ video. object segmentation là nhiệm vụ thu thập thông tin cơ bản, từ đó có thể truy xuất, suy diễn ra các thông tin cần thiết từ đối tượng được phân đoạn. 
% Trong phân đoạn volume, có 2 hướng tiếp cận chính bao gồm: 
% Hướng tiếp cận bằng cách nhìn tổng thể cả một khối object
% something here
% Hướng tiếp cận bằng cách nhìn frame by frame như là nhìn qua hết một chuỗi video. Đối với hướng tiếp cận này 
% - Ưu điểm
% Với chi phí nhẹ, nhìn rõ nét (giữ resolution của ảnh)
% - Nhược điểm 
% Không thể nhìn toàn bộ đối tượng một lần như không gian 3d, mà phải sử dụng mô hình về mặt thời gian. Điều này gây ra khuyết điểm đối với các bộ phận bị chia cắt xa
% - Tiềm năng 
% Với hướng tiếp cận dạng video, cũng khá giống với hướng tiếp cận của một bác sĩ thông thường, bằng cách coi các slide lần lượt. Hướng tiếp cận này explainable và có thể phát triển dựa vào các common behavior dựa trên thói quen hay kiến thức của bác sĩ
% - Challenges:
% Đối với hướng tiếp cận bằng cách coi volume 3d như một video để xử lý các đối tượng tương tác, tắc, biến dạng, mờ chuyển động, sự biến đổi tỷ lệ, việc tracking qua khung hình cũng phải xử lý các thử thách như ngoài tầm nhìn, vật thể nhỏ, đột ngột xuất hiện, vật thể biến mất rồi lại xuất hiện. Mô hình cần đạt được sự thấu hiểu về mặt time-spatial
\subsection{Interactive CT Volume Organ Segmentation}
% interactive scenario là gì?
% Là quy trình gán nhãn - refined được lặp đi lặp lại bán tự động hoá (semi automated) giúp quá trình annotate diễn ra nhanh hơn. Thuật toán sẽ dựa theo đối tượng mà người dùng chú ý để interpolate ra các frame còn lại.
% why interactive?
% Như đã đề cập, việc gán nhãn cho các mô hình tốn chi phí khổng lồ, đặc biệt khan hiếm. Nên việc phát triển các thuật toán cho medical bị hạn chế rất nhiều. Xét ngữ cảnh bán tự động hoá, khi mà thuật toán còn chưa hoàn thiện thì việc gán nhãn bán tự động là cần thiết khi nhân lực expert khan hiếm và khó đào tạo, như cầu tạo ra các bộ dữ liệu 3D rất cần thiết. interactive scenario giúp tăng tôc quá trình gán nhãn như human-in-the-loop. Số nhãn không cần biết trước contribute rất nhiều cho nguồn dữ liệu.
% how to interact?
% Người gán nhãn còn được gọi là annotator, 
% Chọn 1 frame valuable nhất và tiến hành vẽ nguệch ngoạc cho từng đối tượng trong khung này. Dựa trên những nét vẽ nguệch ngoạc này, mô hình phải dự đoán segmentation mask của những đối tượng đó. Mô hình sẽ propagate cho tất cả các slices trong volume. Trong tương tác tiếp theo, annotator chọn một slice độ chính xác kém nhất và cung cấp một tập hợp nguệch ngoạc trong khung này. Những nét vẽ nguệch ngoạc này chỉ ra các positive pixel và negative pixels bằng cách đánh dấu các false positive và false negative.
% Quy trình này được lặp lại cho đến khi người chú thích hài lòng với việc phân đoạn mặt nạ của tất cả các khung.
% Expectation:
% Các phương pháp để giải quyết nhiệm vụ phân đoạn đối tượng volume tương tác phải đáp ứng nhiều mục tiêu, bao gồm nhanh chóng, đánh giá định lượng tốt, tạo ra kết quả ban đầu đủ để thoả mãn user và cải thiện độ chính xác cao sau các lần tương tác.
\lipsum[2]