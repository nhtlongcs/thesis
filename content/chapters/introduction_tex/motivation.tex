\section{Motivations}
\label{sec:motivation}
Segmentation is one of the problems in which the labels for models are scarce because of their high complexity and complication. If we cannot accurately label our data sets, then our models are hardly to be as accurate as we desire. This is a real issue in the medical community, where human resources are often restricted. It is difficult to generate data when we have to prioritize human resources over machine labor. 

The supervised scenario is preferable to the unsupervised scenario in medical data because it is more accurate. The user has to manually annotate the pixel locations of interest with whole volume data. However, the problem occurs that for contiguous frames the difference is not much but the annotator has to time-consuming to perform the annotate action of similar positions on slices with not too different visual features. The unsupervised lack in the low-quality segmentation mask should be limited in terms of results and difficult to put into real life, especially for a field that requires absolute accuracy like medical.

The semi-supervised approach is a great way to take advantage of the benefits of both labeled and unlabeled data. By using this method, we can get more precise quantitative results based on the labeled data. This is important because it allows the model to exploit information from other raw data without any human guide. Additionally, the semi-supervised approach is able to do this while still maintaining accuracy in its results.

Nevertheless, in any scenario above, the process of manually labeling data is slow and inefficient. It can be difficult to reuse previous models, and the adaptability is low. This means that when labeling a new dataset, the process does not take advantage of the power of previous models. As a result, this slows down research and development in the medical field.

Video object segmentation is a pixel-level tracking object through sequence images problem. There are some similarities to when the expert user conducts labeling, although volume data is a 3-dimensional block, when performing labeling, the annotator has to go through each slice and make labeling, but cannot compare. Works conveniently with volume blocks. So now the volume becomes a sequence image and has many similarities with video object segmentation. Interactively volume object segmentation is a problem that has been proposed recently and has some promising directions for development, but it can be leveraged in the use of techniques that can be applied to problems encountered by medical data.

The points made above motivate us to investigate the potential of interactive segmentation with active learning. Such tools would work with people to find valuable positions for refinement and satisfaction of the result. With a certain number of slices, the tool could achieve results that are as good or even better than those produced by annotators.
